% The style of this dictionary is heavily inspired by:
% LaTeX template "Dictionary" by Vel (vel@latextemplates.com)
% https://www.latextemplates.com/template/dictionary
% Licence:
% CC BY-NC-SA 3.0 (http://creativecommons.org/licenses/by-nc-sa/3.0/)
%
% GT LaTeX Dictionary Template
% Author: Giulio Ticli
% Version: 1.0.1 (23 Oct 2023)

\documentclass[10pt, a4paper, twoside]{article}

\usepackage[left=2cm,right=2cm,top=3cm,bottom=2cm,columnsep=20pt]{geometry}
\usepackage[T1]{fontenc}
\usepackage{tipa} % Useful for IPA. https://myweb.uiowa.edu/rsmorris/latex/ipa.html
\usepackage{multicol}

\usepackage{titlesec}
\titleformat*{\section}{\center\bfseries\huge\sffamily}
\usepackage{fancyhdr}
\fancyhead[L]{\textsf{\rightmark}}
\fancyhead[R]{\textsf{\leftmark}}
\fancyhead[C]{\textbf{\textsf{\thepage}}}
\renewcommand{\headrulewidth}{0.4pt}
\fancyfoot{}
\setlength{\parindent}{0pt}
\setlength{\parskip}{6pt}
\setlength{\columnseprule}{0.4pt}

\newcommand*{\entry}[4][]{\markboth{#2}{#2}\textbf{#2}\textsuperscript{#1}\ \textipa{/#3/}\ \textit{#4}\\}
\newcommand{\etymology}[1]{\\\textit{Etymology}\ #1}
\newcommand{\usagenotes}[1]{\\\textit{Usage notes}\ #1}

\usepackage{paralist}
\newenvironment{severalentries}{\begin{inparaenum}[a.]}{\end{inparaenum}}

\begin{document}
    \pagestyle{fancy}

    \section*{Bb}
    \begin{multicols}{2}
        \entry[1]{ball}{bO:l}{n}
        \begin{severalentries}
            \item A hollow or filled sphere.\\
            \item (\textit{Mathematics})\ In a metric space, the set of points whose distance from a given point is less than a given radius.
        \end{severalentries}
        \etymology{From Middle English \textit{bal}, \textit{ball}, \textit{balle}. Ultimately from Proto-Indo-European \textit{*b\textsuperscript{h}el-} (to blow).}

        \entry[2]{ball}{bO:l}{n}
        An elegant dance.

        \entry{barley}{bA:li}{n}
        A cereal.
        \etymology{Sample etymology.}
        \usagenotes{Sample usage notes.}

        \entry{basic}{"beIsik}{adj}
        \begin{severalentries}
            \item Fundamental.\\
            \item Elementary.\\
            \item (\textit{Chemistry})\ Having a pH greater than 7.
        \end{severalentries}
        \etymology{Sample etymology. The quick brown fox jumps over the lazy dog.}
        \usagenotes{Sample usage notes.}

        \entry{beach}{bi:\t{tS}}{n}
        If there is just one definition, you might not want to label it.
        \etymology{Sample etymology. The quick brown fox jumps over the lazy dog.}
        \usagenotes{Sample usage notes.}

        \entry{big}{bIg}{adj}
        My mistress’ eyes are nothing like the sun;
        coral is far more red than her lips’ red:
        if snow be white, why then her breasts are dun;
        if hairs be wires, black wires grow on her head.
        I have seen roses damask’d, red and white,
        but no such roses see I in her cheeks;
        and in some perfumes is there more delight
        than in the breath that from my mistress reeks.
        I love to hear her speak, yet well I know
        that music hath a far more pleasing sound:
        I grant I never saw a goddess go,
        my mistress, when she walks, treads on the ground:
        and yet, by heaven, I think my love as rare
        as any she belied with false compare.
        \etymology{Shakespeare's sonnet 130 used as a placeholder.}
        \usagenotes{Sample usage notes. This time, they're a bit longer than before.}

        \entry{blow}{bl@U}{v}
        \begin{severalentries}
            \item A nice definition.\\
            \item Another definition.
        \end{severalentries}
        \etymology{Sample etymology. The quick brown fox jumps over the lazy dog.}
        \usagenotes{Sample usage notes.}

        \entry{browse}{b\*raUz}{v}
        Multās per gentēs et multa per aequora vectus
        adveniō hās miserās, frāter, ad īnferiās,
        ut tē postrēmō dōnārem mūnere mortis
        et mūtam nēquīquam alloquerer cinerem
        quandoquidem fortūna mihī tētē abstulit ipsum
        heu miser indignē frāter adēmpte mihī
        nunc tamen intereā haec, prīscō quae mōre parentum
        trādita sunt trīstī mūnere ad īnferiās,
        accipe frāternō multum mānantia flētū.
        Atque in perpetuum, frāter, avē atque valē.
        \etymology{Catullus' 101st \textit{carmen} used as a placeholder.}
        \usagenotes{Sample usage notes.}

        \entry{browser}{"b\*raUz@}{n}
        \begin{severalentries}
            \item A nice definition.\\
            \item Another definition.\\
            \item Yet another definition.\\
            \item You can add as many as you want\ldots\\
            \item \ldots unless you run out of letters of the alphabet\ldots\\
            \item \ldots but that won't happen so easily!\\
            \item Just beware not to add a newline after the last item.
            The system already adds one for you.
        \end{severalentries}
        \etymology{Sample etymology.}
        \usagenotes{Sample usage notes.}

        \entry{brunch}{b\*r2n\t{tS}}{n}
        Lorem ipsum dolor sit amet.
        Actually, that's a corrupted text by Cicero.
        \etymology{Sample etymology.}
        \usagenotes{Sample usage notes.}

        \entry[1]{brush}{b\*r2S}{n}
        Check the other \textit{brush}\textsuperscript{1}, too!
        \etymology{Sample etymology.}
        \usagenotes{Sample usage notes.}

        \entry[1]{brush}{b\*r2S}{v}
        \begin{severalentries}
            \item My advice is to retain the same index (1) for two lemmata which are different parts of speech,but share a common etymology (as in this case).\\
            \item Instead, you should switch to a new index if the two lemmata are not related (as in \textit{ball}\textsuperscript{1} and \textit{ball}\textsuperscript{2}).
        \end{severalentries}
        \etymology{Sample etymology. The quick brown fox jumps over the lazy dog.}
        \usagenotes{Sample usage notes. The quick brown dog jumps over the lazy fox.}

        \entry{bubble}{"b2b@l}{n}
        \begin{severalentries}
            \item A nice definition.\\
            \item Another definition.\\
            \item Yet another definition, this time a bit longer (it stretches two lines).\\
            \item You can add as many as you want\ldots\\
            \item \ldots unless you run out of letters of the alphabet\ldots\\
            \item \ldots but that won't happen so easily!\\
            \item Just beware not to add a newline after the last item.
            The system already adds one for you.
        \end{severalentries}
        \usagenotes{Sample usage notes. This time, there's no etymology. Check the header, however!}

        \entry{bucket}{"b2kIt}{n}
        I wandered lonely as a cloud
        that floats on high o'er vales and hills,
        when all at once I saw a crowd,
        a host of golden daffodils;
        beside the lake, beneath the trees,
        fluttering and dancing in the breeze.
        Continuous as the stars that shine
        and twinkle on the Milky Way,
        they stretched in never-ending line
        along the margin of a bay:
        ten thousand saw I at a glance,
        tossing their heads in sprightly dance.
        The waves beside them danced; but they
        out-did the sparkling waves in glee:
        a poet could not but be gay,
        in such a jocund company:
        I gazed---and gazed---but little thought
        what wealth the show to me had brought:
        for oft, when on my couch I lie
        in vacant or in pensive mood,
        they flash upon that inward eye
        which is the bliss of solitude;
        and then my heart with pleasure fills,
        and dances with the daffodils.
        \etymology{A poem by Wordsworth used as a placeholder.}
    \end{multicols}
    \pagebreak
\end{document}